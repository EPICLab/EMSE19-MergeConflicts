%!TEX root = main.tex

\section{How do software developers become \textbf{aware} of merge conflicts? (RQ1)}\label{RQ1}

\boldif{Developers use 2 methods for becoming aware of merge conflicts: proactive and reactive.}
From the \textit{Processes Survey} we found that 29.41\% of participants do not actively monitor for merge conflicts during their development activities.
For the rest of the developers who answered with \emph{yes} or \emph{sometimes} (61.77\%), we identified 61 different tools mentioned in 126 instances.

\subsection{Reactive and Proactive Monitoring for Merge Conflicts}

\boldif{Reactive detection only notifies developers that a merge conflict \emph{has happened}. Developers use it to minimize the size/complexity of the conflict, or it's impact on the team.}

Monitoring for potential merge conflicts can occur at different points in time; before or after a conflict is introduced into a version control system.
Reactive monitoring for merge conflicts involves notifying the developer that a conflict has already occurred.
Although delays in incorporating changes can increase the costs of resolving any subsequent conflicts~\cite{deSouza2003breaking,grinter1995using}, developers still use reactive processes to manage conflicts.
For the developers who answered that they monitor for merge conflicts (replied either \emph{yes} or \emph{sometimes}), we found that 73.68\% (42 out of 57 responses; 6 participants left this field blank) described reactive processes.
For example, one \textit{Processes Survey} participant said they use PagerDuty (an IT incident management and notification system) to detect merge conflicts on important development branches:
\begin{quotation}
	[\ldots] using PagerDuty, we are all notified if a conflict is pushed to the next or future release branches, so that we can respond quickly. We don't want broken code accidentally going out on a release.
\end{quotation}
And another participant mentioned that they try to solve merge conflicts early in order to minimize disruptions to the team:
\begin{quotation}
	We try to catch conflicts early so that fewer developers have to be involved in looking at broken code.
\end{quotation}

\boldif{Proactive monitoring allows devs to detect MC before they happen. However, it is more involved, as it requires a lot more manual effort from the developers.}
Proactive monitoring allows developers to preemptively catch merge conflicts before they happen.
15 participants (14.71\%) mentioned they achieved this by manually tracking incoming changes, such as one participant who indicated:
\begin{quotation}
	I monitor commit logs before I begin merging branches so that I see any potentially overlapping code that will break the merge.
\end{quotation}
Other teams rely more on communication.
This can happen during regular team meetings, to make sure that everybody is aware of each other's tasks, for example another participant said:
\begin{quotation}
	[\ldots] standups allow us to know where everyone is working that week.
\end{quotation}
While 10 participants (9.80\%) indicated that they broadcast their changes in order to notify team members if they will make breaking changes.
One participant indicated that team members:
\begin{quotation}
	[\ldots] send emails before making breaking changes to the API or related sub-modules.
\end{quotation}

When examining the correlation between software development experience and rates of monitoring for merge conflicts (see Table~\ref{monitoring_rates}), we find a moderate negative relationship with weak statistical significance (Pearson correlation coefficient, $r=-0.5962$, $n=6$, $p=0.2117$).
However, examining the data shows that developers with 16--20 years of experience reverse the trend of increased monitoring as development experience increases and thus causes a negative relationship result.
We find similar statistical results when examining across the combinations of monitoring and either proactive or reactive strategies, and conclude that experience holds only moderate effect on the adoption of either proactive or reactive strategies for merge conflict monitoring.

\begin{table}[!htbp]
\renewcommand{\arraystretch}{1.3}
\caption{Rates of Monitoring for Merge Conflicts from \textit{Processes Survey}}
\label{monitoring_rates}
\centering
\begin{tabularx}{\textwidth}{Q{2.2cm}|Z{0.4cm}Z{1cm}Z{0.9cm}Z{1cm}|Z{0.5cm}Z{1cm}Z{0.8cm}Z{1cm}}
\toprule
  \rowcolor[gray]{0.85}
  \parnoteclear % tabularx will otherwise add each note thrice
  Exp.\parnote{Ranges of software development experience.} & 
  \multicolumn{2}{c}{No\parnote{\label{response}Participant responses to the survey question \textit{``do you monitor for merge conflicts?''} (percentages).}} & 
  \multicolumn{2}{c|}{Yes/Sometimes\parnoteref{response}} & 
  \multicolumn{2}{c}{Proactive\parnote{\label{strategy}Participants that monitor for merge conflicts with the proactive or reactive strategy (percentages).}} & 
  \multicolumn{2}{c}{Reactive\parnoteref{strategy}} \\
\midrule
  1--5 years & 12 & (36.4\%) & 21 & (63.6\%) & 5 & (23.8\%) & 16 & (76.2\%)\\
  \rowcolor[gray]{0.95}6--10 years & 8 & (29.6\%) & 19 & (70.4\%) & 8 & (42.1\%) & 11 & (57.9\%)\\
  11--15 years & 4 & (23.5\%) & 13 & (76.5\%) & 5 & (38.5\%) & 8 & (61.5\%)\\
  \rowcolor[gray]{0.95}16--20 years & 5 & (41.7\%) & 7 & (58.3\%) & 2 & (28.6\%) & 5 & (71.4\%)\\
  21--25 years & 1 & (25.0\%) & 3 & (75.0\%) & 1 & (33.3\%) & 2 & (66.7\%)\\
  \rowcolor[gray]{0.95}26+ years & 0 & (\%) & 0 & (0.0\%) & 0 & (0.0\%) & 0 & (0.0\%)\\
\bottomrule
\end{tabularx}
\parnotes
\end{table}

\boldif{Developers do not regularly actively monitor for merge conflicts. ...}
To conclude, only a third of developers actively monitor for merge conflicts.
When developers are caught unaware of the conflict, additional developers and resources are necessary to fix the conflicting code.
This can lead to more frustration, as they do not have any warning of when the conflict will occur and whether they have the time to deal with it immediately.

\subsection{Tools for Monitoring for Merge Conflicts}

\boldif{The tools that developers use allow for only a \emph{reactive} approach.}
Examining the tools used by participants with reactive processes, we find that 87.72\% of these participants rely on version control systems (e.g. Git, SVN, TFS, CVS), and 21.05\% use continuous integration systems (e.g. Jenkins, Travis CI, TeamCity).
Table~\ref{s1_toolset} presents the top 10 tools developers use when monitoring for merge conflicts, including the totals for both reactive and proactive strategies.

Additionally, we examine the tools used by participants with proactive processes.
We find that all participants with a proactive strategy rely on version control systems, and 33.33\% use continuous integration systems.
Additionally, 26.66\% of proactive participants use code analysis tools (e.g. SonarQube, Code Climate).

We find that the majority of tools used by developers for merge conflict monitoring are built to only support reactive strategies, and that multiple tools must be used in conjunction for a proactive approach.

%\boldif{Devs do not use existing workspace awareness tools that come from the acadmemia.}
%When collaborating, developers generally rely on passive communication tools, like email, to coordinate.
%Developers are currently not leveraging the functionalities provided by many research prototypes (e.g., Palant\'{i}r~\cite{palantir}, Crystal~\cite{Brun2011}) that are specifically designed to facilitate proactive conflict detection.

\newcolumntype{b}{X}
\newcolumntype{s}{>{\hsize=8\hsize}X}

\begin{table}[!htbp]
\renewcommand{\arraystretch}{1.3}
\caption{Merge Awareness Toolsets (Top 10) from \textit{Processes Survey}}
\label{s1_toolset}
\centering
\begin{tabularx}{\textwidth}{Q{2.51cm}l|RQ{1.11cm}RQ{1.12cm}|RZ{1cm}}
\toprule
  \rowcolor[gray]{0.85}
  \parnoteclear % tabularx will otherwise add each note thrice
  Tool\parnote{\textit{Processes Survey} participants were allowed to provide multiple tools. 57 out of 102 participants (56\%) indicated the use of at least one merge awareness tool.} & Description & \multicolumn{2}{c}{Proactive\parnote{\label{tool_responses}Participants using this tool with the proactive or reactive strategy (percentages).}} & \multicolumn{2}{c}{Reactive\parnoteref{tool_responses}} & \multicolumn{2}{|c}{Total\parnote{Total number of survey participants using each particular tool.}}\\
\midrule
  Git & Version Control System & 10 & (9.8\%) & 30 & (29.4\%) & 40 & (39.2\%)\\
  \rowcolor[gray]{0.95}GitHub & Project Hosting Site & 2 & (2.0\%) & 5 & (4.9\%) & 7 & (6.9\%)\\
  Email (generic) & Email Client/System & 2 & (2.0\%) & 4 & (3.9\%) & 6 & (5.9\%)\\
  \rowcolor[gray]{0.95}SVN\parnote{Subversion} & Version Control System & 0 & (0.0\%) & 4 & (3.9\%) & 4 & (3.9\%)\\
  VCS (generic) & Version Control System & 2 & (2.0\%) & 2 & (2.0\%) & 4 & (3.9\%)\\
  \rowcolor[gray]{0.95}Visual Studio & IDE & 1 & (1.0\%) & 2 & (2.0\%) & 3 & (2.9\%)\\
  PagerDuty & IT Incident Mgmt. & 0 & (0.0\%) & 3 & (2.9\%) & 3 & (2.9\%)\\
  \rowcolor[gray]{0.95}GitLab & Project Hosting Site & 2 & (2.0\%) & 1 & (1.0\%) & 3 & (2.9\%)\\
  Jenkins & Continuous Integration & 0 & (0.0\%) & 3 & (2.9\%) & 3 & (2.9\%)\\
  \rowcolor[gray]{0.95}TFS\parnote{Team Foundation Server} & Version Control System & 1 & (1.0\%) & 1 & (1.0\%) & 2 & (2.0\%)\\
\bottomrule
\end{tabularx}
\parnotes
\end{table}


\boldif{Therefore their approaches are mostly \emph{reactive,} and their tool selection reflects that.}
To summarize, we find that developers employ \emph{reactive} processes, even if they are proactive in monitoring for merge conflicts once they have occurred.
This can be seen as a consequence of the tools that developers have at their disposal.
All the tools mentioned focus support exclusively towards a \emph{reactive} approach, which biases developers towards one particular solution.
If developers want a more \emph{proactive} approach, then based on the tools they use, they need to come up with their own solution.
The most often cited techniques involve increasing communication among developers.
While this technique might be effective in small teams, it scales very poorly and cannot be effectively used in larger organizations~\cite{brooks1974mythical}.

\boldif{All the above point towards a need for better collaborative tools, that promote a proactive approach}
Finally, our results point to the conclusion that developers are not implementing proactive concepts shown in research prototypes (e.g. Palant\'{i}r~\cite{sarma_palantir:_2003} and Crystal~\cite{Brun2011}), and are therefore not leveraging those tools to actively monitor for merge conflicts.
However, developers are trying to mitigate the severity of merge conflicts by attempting to resolve them as soon as they become aware.
