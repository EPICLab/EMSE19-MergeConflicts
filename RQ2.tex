%!TEX root = main.tex

\subsection{\textbf{RQ2:} What difficulties do software developers experience with merge conflicts?}\label{RQ2}

\begin{table}[!htbp]
\renewcommand{\arraystretch}{1.2}
\caption{Factors of Merge Conflict Difficulty from Barriers Survey (S2)}
\label{s2_factors}
\centering
\begin{tabularx}{\textwidth}{>{\rowmac}c | >{\rowmac}l | *1{>{\rowmac}c} | *2{>{\rowmac}c}<{\clearrow}}
\toprule
  \parnoteclear % tabularx will otherwise add each note thrice
  Factor & Description & \likertscale{1,2,3,4,5} & Median\parnote{Responses on 5-point Likert scale indicating the degree of effect on resolution difficulty (1 indicates \textit{no effect}, 5 indicates \textit{great effect}).} & Mean\textsuperscript{i} \\
\midrule
  \setrow{\bfseries}F1 & Complexity of conflicting lines of code & \likertplot{coordinates {(1,5)(2,29)(3,38)(4,56)(5,34)}}{28.2}{5,29,38,56,34} & 4 & 3.52 \\
  \setrow{\bfseries}F2 & Expertise in area of conflicting code & \likertplot{coordinates {(1,5)(2,23)(3,50)(4,54)(5,30)}}{28.2}{5,23,50,54,30} & 4 & 3.50 \\
  \setrow{\bfseries}F3 & Complexity of files with conflicts & \likertplot{coordinates {(1,8)(2,34)(3,49)(4,51)(5,18)}}{28.2}{8,34,49,51,18} & 3 & 3.23 \\
  \setrow{\bfseries}F4 & Number of conflicting lines of code & \likertplot{coordinates {(1,2)(2,40)(3,64)(4,45)(5,11)}}{28.2}{2,40,64,45,11} & 3 & 3.14 \\
  F5 & Time to resolve a conflict & \likertplot{coordinates {(1,14)(2,56)(3,51)(4,25)(5,15)}}{28.2}{14,56,51,25,15} & 3 & 2.82 \\
  F6 & Atomicity of changesets in conflict & \likertplot{coordinates {(1,20)(2,48)(3,51)(4,29)(5,13)}}{28.2}{20,48,51,29,13} & 3 & 2.80 \\
  F7 & Dependencies of conflicting code & \likertplot{coordinates {(1,20)(2,56)(3,39)(4,33)(5,14)}}{28.2}{20,56,39,33,14} & 3 & 2.78 \\
  F8 & Number of files in the conflict & \likertplot{coordinates {(1,10)(2,69)(3,50)(4,26)(5,6)}}{28.2}{10,69,50,26,6} & 3 & 2.68 \\
  F9 & Non-functional changes in codebase & \likertplot{coordinates {(1,47)(2,63)(3,31)(4,15)(5,4)}}{28.2}{47,63,31,15,4} & 2 & 2.16 \\
\bottomrule
\end{tabularx}
\parnotes
\end{table}

To understand the perspective of software developers when encountering a merge conflict, we asked interview participants to reflect on situations when they initially face a merge conflict: what kind of information do they seek, how do they approach the resolution of the conflict, and what tools do they use. 

We identified nine factors (from card sorting) that developers consider when approaching a conflict and attempting to determine its difficulty (see Table~\ref{s2_factors}). 
We asked survey participants to rate how each of these nine factors affected their perceptions of difficulty when approaching a merge conflict.

We received 162 responses and present the aggregated results in Table~\ref{s2_factors}; ranked according to the mean score for each factor.
Here, we discuss in detail the top 4 factors with a mean score greater than $3.00$.
These factors can be grouped into themes of \textit{technical aspects} and \textit{expertise}, and our results are presented according to these groups.

\subsubsection{Technical Aspects}\label{artifact-based-factors}
Two of the top four factors refer to perceptions about the complexity of merge conflicts (F1, F3), with the third factor being \textit{number of conflicting lines of code} (F4), which can be construed as a specific metric of complexity of the conflict. 
While developers mentioned complexity of the lines of code and the file, none mentioned using metrics, such as cyclomatic complexity~\cite{fenton2000quantitative}\cite{mccabe1976complexity} or Function Point Analysis~\cite{garmus2001fpa}\cite{symons1988function}. 
Instead, developers made educated guesses on the complexity of the code based on their own experience of either writing the code, or having worked with it. 
Some of the simple to compute metrics, such as \textit{number of conflicting lines of code} (F4), \textit{number of files in the conflict} (F8), \textit{atomicity of changesets in conflict} (F6), and the \textit{time to resolve a conflict} (F5) were mentioned. 
The only factor where static analysis tools can help was in identifying the \textit{dependencies of conflicting code} (F7).
This indicates that understanding the complexity of the conflicting code is important, but developers do not use the metrics that have been proposed by research.
While some of the simple proxies for complexity are used, developers primarily rely on their own ``judgement" of the complexity of the conflict.

This perception of the conflict complexity can affect whether a developer resolves the conflict immediately (when small), or whether they should wait to examine the conflict when further resources are available; P8 commented:
\begin{quoting}
\textit{``Small is always easy. A 1-line merge conflict is always easier to resolve than a 400-line merge conflict, and can be done now.''}
\end{quoting}

If a merge conflict is perceived to be large or complex, a developer may decide to forgo attempting to resolve it through code manipulation and choose to revert the changes instead~\cite{Guzzi2015}.
This ``nuclear option'' requires developers to disrupt the development flow, set aside their current development work, and potentially remove ``good'' code that was not part of the conflict in order to return to a non-conflicting state.
In the interview, P1 describes this process as:
\begin{quoting}
\textit{``If you have many conflicts involved, many commits in the conflict...throw one of the branches away. You cannot combine tens of commits conflicting...it's not sane!''}
\end{quoting}

Further, when integrators are preparing code for production environments they prioritize merge conflicts for code review based upon the perceived difficulty of resolving the affected code.
We find that these decisions rely on human judgement factors as much as they rely on data-driven metrics.
Developers may not have the time to compute project-wide complexity metrics, such as those proposed in  literature.
Therefore, we need metrics that can be easily calculated by ``lay developers" as they face a conflict. 
%are human-aware and take into account the perceived difficulties of merge conflicts.

\subsubsection{Expertise}\label{knowledge-based-factors}
Our findings show that \textit{expertise in the area of conflicting code}~(F2) is one of the top factors in determining the difficulty of a merge conflict. This reiterates the fact that developers rely on their own knowledge about the conflicting codebase when approaching a conflict. 

Our results indicate that when developers feel they don't have the expertise in the conflicting codebase, they consider the conflict difficult to merge and seek out more information or assistance from others. 
P5 illustrated this need for expertise when describing his workflow: 
\begin{quoting}
	\textit{``A lot of what I work on is in my own little area...I know what to do\textellipsis But in [unfamiliar part of code], then I'll get someone else to resolve the merge conflict for me. It's someone else's code, and I don't want to screw it up.''}
\end{quoting}

Our findings confirm the need for tools that identify appropriate experts~\cite{CostaSarma} and encourage further research into selection of knowledgeable developers for merge conflict resolution.

%%%%%%%%%%%%%%%%%%%%%%%%%%%%%%%%%%%%%%%%%%%%%%%%%%%%%%%%%%%%%%%%%

%\subsubsection{Interviews}
%%The interview results suggest that developers approach merge conflicts...
%
%\subsubsection{Survey}
%Our survey suggests that regardless of gender, developer role, experience level, project size, and source distribution model, software practitioners say that the following factors affect the difficulty of a merge conflict most: 
%\begin{itemize}
%\item \textit{Complexity of conflicting lines of code}
%\item \textit{Your knowledge/expertise in area of conflicting code}
%\end{itemize}
%
%Similarly, software practitioners across every measured demographic perceived the following factors to be less important when deciding the difficulty of a merge conflict:
%\begin{itemize}
%\item \textit{Non-functional changes (whitespace, renaming, etc)}
%\item \textit{Number of files in the conflict}
%\end{itemize}
%
%While survey participants did not agree that non-functional changes strongly factor into the difficulty of a merge conflict, it is still worth noting that several interview participants named non-functional changes, such as large refactor or reformatting changes, as a cause for merge conflicts. This suggests that non-functional changes may increase the likelihood of a merge conflict happening, but they do not contribute to the conflict's difficulty.
%
%However, some demographics do view certain difficulties. For instance, open-source developers think that \textit{Atomicity of change sets in the conflict} impacts the difficulty, while closed-source developers and people who split their time evenly think that atomic change sets have no effect on the difficulty. This may be explained by the findings in Rigby et al\cite{OSS_smaller_commits}, which shows that open-source projects tend to review smaller changes than closed-source projects because "The small size lets reviewers focus on the entire change, and the incrementality reduces reviewers’ preparation time and lets them maintain an overall picture of how the change fits into the system." It is possible that our result reflects this difference of culture.
%
%We also found that Project Maintainers say that \textit{Time to resolve a conflict} has an effect, while no other role agrees. This suggests that those in a maintainer role may be more subject to time-related constraints such as maintenance or release schedules. 
%
%\comment{Project Managers say no effect because they focus on project schedules, not conflict resolutions, i.e. they are higher level/abstraction?}
%
%\todo{might be previous work}
%Support and infrastructure roles (System Engineer, Sys Admin, System Architect, DevOps) emphasized that \textit{Dependencies of conflicting code on other components} have more of an effect than other roles did. This might be due to infrastructure systems being maintained in a live environment, or systems that are currently in production use and at risk of real-time dependency failures. 
%
%Developers on projects of size 1 say that \textit{Dependencies of conflicting code on other components}. Because no other project sizes agree with this idea, we hypothesize that this could be due to their high dependence on external code because of the software production limitations of a 1-developer team.
%
%We also found that the group of developers with 21-25 years of experience frequently contradicted general consensus, but it seems more likely that these differences were simply due to the group's small sample size (4).

%We asked participants how much they trust their merging, history exploration, and/or conflict resolution tools, and 57.9\% of participants reported that they trusted these tools either \textit{A Lot} or \textit{Completely}. While this is a majority of developers, this still leaves a significant number of people (42.1\%) who trust their tools \textit{A moderate amount} or \textit{A little}. Though we had the option for \textit{Not at all}, no participants selected this option, presumably because users stop using tools that they do not trust at all. While we found no previous work discussing the threshold for how much users must trust tools for a good tool experience, we postulate that users who cannot trust their tools \textit{A Lot} or \textit{Completely} will avoid relying on such tools too much.