\begin{abstract}
Merge conflicts occur when developers make concurrent changes to the same part of the code.
They are an inevitable and painful aspect of collaborative software development.
Because of that tool builders and researchers have focused on the prevention and automatic resolution of merge conflicts.
However, there is little empirical knowledge about how developers actually approach and perform merge conflict resolutions.
Without such knowledge, tool builders might be building on the wrong assumptions and researchers might miss opportunities for improving the state of the art.

We conducted semi-structured interviews of 10 software developers across 7 organizations, including both open-source and commercial projects.
We identify key processes, techniques and perceptions from developers, which we extend and validate via a two surveys, of 102 developers and 162 developers.

We find that developers are directly impacted by their perception of the complexity of the conflicting code, and may alter the timeline in which to resolve these conflicts, as well as the methods employed for conflict resolution based upon that initial perception.
Developers' perceptions alter the impact of tools and processes that have been designed to preemptively and efficiently resolve merge conflicts.
Understanding whether developers will react according to standard use cases is important when creating human-oriented tools to support development processes.
\end{abstract}