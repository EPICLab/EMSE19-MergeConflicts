\begin{abstract}
Merge conflicts occur when developers make concurrent changes to the same part of the code.
They are an inevitable and painful aspect of collaborative software development.
Because of that tool builders and researchers have focused on the prevention and automatic resolution of merge conflicts.
However, there is little empirical knowledge about how developers actually approach and perform merge conflict resolutions.
Without such knowledge, tool builders might be building on the wrong assumptions and researchers might miss opportunities for improving the state of the art.

We conducted semi-structured interviews of 10 software developers across 7 organizations, including both open-source and commercial projects.
We identify key processes, techniques, and perceptions from developers, which we extend and validate via two surveys, of 102 developers and 162 developers.

Developers rely on reactive strategies of monitoring for merge conflicts.
Without proactive monitoring, developers must interrupt development in order to resolve the conflict.
However, we find that developers defer responding to conflicts based on their perception of the complexity of conflicting code and that these defers effect the entire team.
Developers also rely on this perception to evaluate their merge conflict resolutions.
They do this in conjunction with other criteria, such as tests and successful compilations.
Finally, developers' perceptions alter the impact of tools and processes that have been designed to preemptively and efficiently resolve merge conflicts.
Understanding their processes and perceptions can help design human-oriented tools that better support their individual development processes.
\end{abstract}