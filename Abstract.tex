\begin{abstract}
Merge conflicts occur when developers make concurrent changes to the same part of the code.
They are an inevitable and disruptive aspect of collaborative software development.
Thus tool builders and researchers have focused on the prevention and automatic resolution of merge conflicts.
However, there is little empirical knowledge about how developers actually monitor for merge conflicts and plan, perform, and evaluate resolutions.
Without such knowledge, tool builders might be building on the wrong assumptions and researchers might miss opportunities for improving the state of the art.

We conducted semi-structured interviews of 10 software developers across 7 organizations, including both open-source and commercial projects.
We identify key processes, techniques, and perceptions from developers, which we extend and validate via two surveys, a \emph{Processes Survey} and a \emph{Barriers Survey,} of 102 and 162 developers, respectively.

Among others, we found that developers rely on reactive strategies of monitoring for merge conflicts.
Without proactive monitoring, developers must interrupt development in order to resolve the conflict.
However, we find that developers defer responding to conflicts based on their perception of the complexity of the conflicting code and that deferring affects the entire team.
Developers also rely on this perception to evaluate their merge conflict resolutions.
They do this in conjunction with criteria such as tests and successful compilations.
Finally, developers' perceptions alter the impact of tools and processes designed to preemptively and efficiently resolve merge conflicts.
Understanding their processes and perceptions can help design human-oriented tools that better support their individual development processes.
\end{abstract}