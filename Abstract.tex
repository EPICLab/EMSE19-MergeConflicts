\begin{abstract}
Merge conflicts occur when software developers need to work in parallel and are inevitable in software development.
Tool builders and researchers have focused on the prevention and resolution of merge conflicts, but there is little empirical knowledge about how developers actually approach and perform merge conflict resolution.
Without such knowledge, tool builders might be building on wrong assumptions and researchers might miss opportunities for improving the state of the art.

We conducted semi-structured interviews of 10 software developers across 7 organizations, including both open-source and commercial projects.
We identify key processes and techniques from developers, which we extend and validate via a survey of 102 developers.
We also identify the key concepts and perceptions from developers in the interviews, which we validate via an additional survey of 162 developers.

We find that developers are directly impacted by their perception of the complexity of the conflicting code, and may alter the timeline in which to resolve these conflicts, as well as the methods employed for conflict resolution based upon that initial perception.
Developers' perceptions alter the impact of tools and processes that have been designed to preemptively and efficiently resolve merge conflicts.
Understanding whether developers will react according to standard use cases is important when creating human-oriented tools to support development processes.
\end{abstract}